%
% Example based on documentation for the songs project:
% http://songs.sourceforge.net/
%
\documentclass[14pt]{extarticle}%TUTAJ WIELKOŚĆ

\usepackage[chorded]{songs}
\usepackage[margin=1.5cm,footskip=0.25in,headheight=34.5pt]{geometry}
\usepackage{polski}
 \usepackage{lipsum}
 \usepackage{microtype}
 \usepackage{vwcol} 
 \usepackage{hyperref}
\hypersetup{
    linktoc=all,     %set to all if you want both sections and subsections linked
    linkcolor=blue,  %choose some color if you want links to stand out
}


  
\begin{document}
\newcommand{\pios}[3]{
\section{
#1
}
\noindent
\begin{minipage}[t]{
0.7\textwidth
}
#2
\end{minipage}%
\begin{minipage}[t]{0.3\textwidth}
#3
\end{minipage}%
}
\newcommand{\reflyrics}[1]{
\hspace*{1cm}%Refren
\begin{minipage}{.8\textwidth}%
#1
\end{minipage}%
}
\newcommand{\refchords}[1]{\hspace*{-1cm}
\begin{minipage}{.8\textwidth}%REFREN
#1

\end{minipage}%
}

\tableofcontents

\section{
Pożegnanie Liverpoolu
}
\noindent
\begin{minipage}[t]{
0.7\textwidth
}
1. Żegnaj nam dostojny, stary porcie,

Rzeko Mersey żegnaj nam! 

Zaciągnąłem się na rejs do Kalifornii, 

Byłem tam już niejeden raz. 
\\\\
\hspace*{1cm}%Refren
\begin{minipage}{.8\textwidth}%
Ref.: A więc żegnaj mi, kochana ma! 

Za chwilę wypłyniemy w długi rejs. 

Ile miesięcy Cię nie będę widział, 

Nie wiem sam, 

Lecz pamiętać zawsze będę Cię.
\end{minipage}%
\\

2. Zaciągnąłem się na herbaciany kliper,

Dobry statek, choć sławę ma złą,

A że kapitanem jest tam stary Burgess,

Pływającym piekłem wszyscy go zwą.\\


3. Z kapitanem tym płynę już nie pierwszy raz,

Znamy się od wielu, wielu lat.

Jeśliś dobrym żeglarzem – radę sobie dasz,

Jeśli nie – toś cholernie wpadł.\\

4. Żegnaj nam dostojny, stary porcie,

Rzeko Mersey żegnaj nam.

Wypływamy już na rejs do Kalifornii,

Gdy wrócimy – opowiemy wam.
\\\\
\hspace*{1cm}%\\
\begin{minipage}{.8\textwidth}%
Ref.: A więc żegnaj mi, kochana ma! 

Za chwilę wypłyniemy w długi rejs. 

Ile miesięcy Cię nie będę widział, 

Nie wiem sam, 

Lecz pamiętać zawsze będę Cię. 
\end{minipage}%
\end{minipage} % no space if you would like to put them side by side
\begin{minipage}[t]{0.3\textwidth}
C C7 F C

C G

C C7 F C

C G C
\\

\hspace*{-1cm}%
\begin{minipage}{.8\textwidth}%
G F C

C G

C C7

F C 

C G7 C
\end{minipage}%

\end{minipage}

% Niezweryfikowane

\pios{Bitwa}{
Okręt nasz wpłynął w mgłę i fregaty dwie \\
Popłynęły naszym kursem by nie zgubić się. \\
Potem szkwał wypchnął nas poza mleczny pas \\
I nikt wtedy nie przypuszczał,\\
że fregaty śmierć nam niosą. \\

\reflyrics{
Ref.: Ciepła krew poleje się strugami,\\
Wygra ten, kto utrzyma ship. \\
W huku dział ktoś przykryje się falami, \\
Jak da Bóg, ocalimy bryg. \\
}

Nagły huk w uszach grał i już atak trwał,\\
To fregaty uzbrojone rzędem w setkę dział.\\
Czarny dym spowił nas, przyszedł śmierci czas,\\
Krzyk i lament mych kamratów, przerywany ogniem katów.\\
\\
Pocisk nasz trafił w maszt, usłyszałem trzask,\\
To sterburtę rozwaliła jedna z naszych salw.\\
„Żagiel staw” krzyknął ktoś, znów piratów złość,\\
Bo od rufy nam powiało, a fregatom w mordę wiało.\\
\\
Z fregat dwóch tylko ta pierwsza w pogoń szła,\\
Wnet abordaż rozpoczęli, gdy dopadli nas.\\
Szyper ich dziury dwie zrobił w swoim dnie,\\
Nie pomogło to psubratom, reszta z rei zwisa za to.\\
\\
Po dziś dzień tamtą mgłę i fregaty dwie,\\
Kiedy noc zamyka oczy, widzę w moim śnie.\\
Tamci, co śpią na dnie, uśmiechają się,\\
Że ich straszną śmierć pomścili bracia, którzy zwyciężyli.
}{
 e D C a\\
 e D G H7\\
 e D C a\\
 \\
 e D G H\\

\refchords{
G D e h\\
C D e\\
G D e h\\
C D e
}

}%USING CUSTOM COMMANDS
% Niezweryfikowane

\pios{Biała sukienka}{
Czasami, gdy mam chandrę i jestem sam, \\
Kieruję wzrok za okno, wysoko tam, \\
Gdzie nad dachami domów i w noc, i dniem, \\
Nadpływa kołysząca, …marzeniem, …snem. \\

\reflyrics{
    I ona taka w tej białej sukience, \\
    Jak piękny ptak, który zapiera w piersi dech. \\
    Chwyciłem mocno jej obie ręce \\
    Oczarowany, zasłuchany w słodki śmiech. \\
    
    I cała w żaglach, jak w białej sukience, \\
    Jak piękny ptak, który zapiera w piersi dech. \\
    Chwyciłem mocno ster w obie ręce \\
    I żeglowałem zasłuchany w fali śpiew. \\
}

Wspomnienia przemijają, a w sercu żal, \\
Wciąż w łajbę się przemienia dziewczęcy czar. \\
Jeżeli mi nie wierzysz, to gnaj co tchu, \\
Tam z kei możesz ujrzeć coś z mego snu. \\

Nie wiem, czy jeszcze kiedyś zobaczę ją,  \\
Czy tylko w moich myślach jej oczy lśnią?  \\
Gdy pochylona, ostro do wiatru szła…  \\
Znowu się przeplatają obrazy dwa:  \\

}{
a e F C \\
a e F G C \\
E a G7 D \\
a e F G C \\

\refchords{
    C G \\
    C F C \\
    G C F \\
    C a D7 G \\
}
}
% Niezweryfikowane

\pios{Bijatyka (24. lutego)}{
To dwudziesty czwarty był lutego, \\
Poranna zrzedła mgła, \\
Wyszło z niej siedem uzbrojonych kryp, \\
Turecki niosły znak. \\

\reflyrics{
    No i znów bijatyka, \\
    no i znów bijatyka, \\
    no i bijatyka cały dzień, \emph{(dzień! dzień!)} \\
    I porąbany dzień, i porąbany łeb, \\
    Razem bracia, aż po zmierzch! \\
}

Już pierwszy skrada się do burt, \\
A zwie się „Goździk” i \\
Z Algieru Pasza wysłał go, \\
Żeby nam upuścił krwi. \\

Następny zbliża się do burt, \\
A zwie się „Róży Pąk”. \\
Plunęliśmy ze wszystkich rur – \\
Bardzo prędko szedł na dno. \\

W naszych rękach dwa i dwa na dnie, \\
Cała reszta zwiała gdzieś, \\
No a jeden z nich zabraliśmy \\
Na starej Anglii brzeg. \\

}{
G \\
G \\
e G \\
C D e \\

\refchords{
    G \\
    G \\
    G D \\
    e G \\
    C D e \\
}
}

\end{document}